%\documentclass{sig-alternate-05-2015}
 \documentclass[10pt,conference]{IEEEtran}
 
\usepackage{url}
\usepackage[table,xcdraw]{xcolor}
\usepackage{listings}
\usepackage{eurosym}
\usepackage{amsfonts}
\usepackage{balance}
\usepackage{cite} %this package is awesome - it reorders lists of citations to be in numeric order
\usepackage{pifont}
\newcommand{\xmark}{\ding{53}}%
\usepackage{eqparbox}
\usepackage{hyperref}
\usepackage{subfigure}
%\usepackage{fancyvrb}

\usepackage{algorithm}% http://ctan.org/pkg/algorithms
\usepackage{algpseudocode}% http://ctan.org/pkg/algorithmicx
\newcommand{\var}[1]{{\ttfamily#1}}% variable

% Tables
\usepackage{booktabs}
\usepackage{pbox}
\renewcommand{\arraystretch}{1.2}
\usepackage{arydshln}
%\renewcommand*\cmidrule{} % No middle lines
%\renewcommand{\arraystretch}{1.5} % Additional spacing with no middle lines
%\renewcommand*\cmidrule{\hdashline[1pt/2pt]}% Dashed middle lines
\renewcommand*\cmidrule{\midrule[0.001em]} % Thin middle lines
%\renewcommand*\cmidrule{\midrule} % Thick middle lines
\newcommand{\todoNow}[1]{\textbf{\textcolor{red}{TODO.NOW: #1}}} %comment out for submission
\newcommand{\todoMid}[1]{\textbf{\textcolor{magenta}{TODO.MID: #1}}} %comment out for submission
%\newcommand{\todoMid}[1]{\textbf{\textcolor{magenta}{}}} %comment out for submission
\newcommand{\todoLast}[1]{\textbf{\textcolor{blue}{TODO.LAST: #1}}} %comment out for submission
\newcommand{\clarify}[1]{{\color{blue}\{CLARIFY: #1\}}}

\usepackage{tikz}
\def\checkmark{\tikz\fill[scale=0.4](0,.35) -- (.25,0) -- (1,.7) -- (.25,.15) -- cycle;}



%Images
%\usepackage[pdftex]{graphicx}
\DeclareGraphicsExtensions{.pdf,.jpg,.png}

\hyphenation{second-ly ap-pen-dix}

\clubpenalty = 10000
\widowpenalty = 10000
\displaywidowpenalty = 10000

\newcommand{\horiz}{\hspace{2.1pt}}
\renewcommand{\topfraction}{.9}

\newcommand{\ignore}[1]{}

\begin{document}
%\bstctlcite{IEEEexample:BSTcontrol}
%
% paper title
% can use linebreaks \\ within to get better formatting as desired
\title{Understandability Smells in Regular Expressions}


\author{
\IEEEauthorblockN{Carl Chapman}
\IEEEauthorblockA{Department of Computer Science\\
Iowa State University\\
carl.chapman@gmail.com}
\and 
\IEEEauthorblockN{Kathryn T. Stolee}
\IEEEauthorblockA{Department of Computer Science\\
North Carolina State University\\
ktstolee@ncsu.edu}  
}


%\numberofauthors{2}
%\author{
%% 1st. author
%\alignauthor
%Carl Chapman\footnote{* work performed while at Iowa State University}\\
%       \affaddr{Sandia National Labs}\\
%       \affaddr{NM}\\
%       \email{carl.chapman@gmail.com}
%\alignauthor
%Kathryn T. Stolee\\
%       \affaddr{Department of Computer Science}\\
%       \affaddr{North Carolina State University}\\
%       \email{ktstolee@ncsu.edu}
%\alignauthor
%}


\maketitle


\begin{abstract}
Regular expressions (regexes) are powerful tools employed across many tasks and platforms. Regexes can be very complex and prior work has shown that developers find regexes to be difficult to compose and understand. Due to a rich feature set, there is more than one way to compose a regex to get the same desired behavior. With the goal of identifying code smells that impact comprehension, we conducted an empirical study with 180 participants and 35 pairs of behaviorally equivalent but syntactically different regexes, and evaluate the understandability of various regex language features. We found that, for example,  patterns requiring one or more  repetitions of a character are more understandable when expressed using the plus (e.g., \verb!`:+'!) operator than the kleene star operator   (e.g., \verb!`::*'!). We further analyze regexes in GitHub to find community standards, or common usages of various features. Finally, we identify smelly and non-smelly representations based on a combination of community standards and understandability metrics, and form recommendations on how to transform regexes to enhance comprehension and conformance to community standards. 

%Regular expressions (regexes) are powerful tools employed across many tasks and platforms. Regexes can be very complex and prior work has shown that developers find regexes to be difficult to compose and understand. Due to a rich feature set, there is more than one way to compose a regex to get the same desired behavior. With the goal of finding potential transformations that improve regex comprehension, we conducted an empirical study with 180 participants and 36 pairs of behaviorally equivalent but syntactically different regexes, and evaluate the understandability of various regex language features. We found that, for example,  patterns requiring one or more  repetitions of a character are more understandable when expressed using the plus (e.g., `:+') operator than the kleene star operator   (e.g., `::*'). We further analyze regexes in GitHub to find community standards, or common usages of various features. Finally, we identify preferred representations based on a combination of community standards and understandability metrics, and form recommendations on how to transform regexes to enhance comprehension. 


\end{abstract}

\section{Introduction }

Regular expressions are used frequently by developers for many purposes, such as parsing files, validating user input, or querying a database.
Regexes are also employed in MySQL injection prevention~\cite{Yeole:2011:ADT:1980022.1980229} and network intrusion detection~\cite{network}. 
However, recent research has suggested that regular expressions  are hard to understand, hard to compose, and error prone~\cite{Spishak:2012:TSR:2318202.2318207}.
Given the difficulties with working with regular expressions and how often they appear in software projects and processes, it seems fitting that efforts should be made to ease the burden on developers.

Tools have been developed to make regexes easier to understand, and many are freely available.
Some tools will, for example, highlight parts of regex patterns that match parts of strings to aid in comprehension.\footnote{\url{https://regex101.com/}}
Others will automatically generate strings that are matched by the regular expessions~\cite{hampi} 
or  automatically generate regexes when given a list of strings to match~\cite{Babbar:2010:CBA:1871840.1871848, Li:2008:REL:1613715.1613719}.
The commonality of such tools provides evidence that people need help with regex composition and understandability.

In software engineering, code smells have been found to hinder understandability of source code~\cite{abbes2011empirical, du2006does}.
Once removed through refactoring, the code becomes more understandable, easing the burden on the programmer.
In regular expressions, such code smells have not yet been defined, perhaps in part because it is not clear what makes a regex difficult to understand or maintain. 

As with source code, in regular expressions, there are multiple ways to express the same semantic concept.
For example, the regex, \verb!`aa*'! matches an \verb!a! followed by zero or more \verb!a!'s, and is is equivalent to \verb!`a+'! , which matches one or more \verb!a!'s.
What is not clear is which representation,  \verb!`aa*'!  or  \verb!`a+'!, is more easily understood.
%Preferences in regex refactorings could vary, including which is easier to maintain, easier to understand, or better conforms to community standards, depending on the goals of the programmer.

In this work, we aim to identify  comprehension smells in regular expressions. 
We  identify equivalence classes of regex representations that provide options for representing: double-bounds in repetitions (e.g., \verb!`a{1,2}'! or \verb!`a|aa'!), single-bounds in repetitions (e.g., \verb!`a{2}'! or \verb!`aa'!), lower bounds in repetitions (e.g., \verb!`a{2,}'! or \verb!`aaa*'!), character classes (e.g., \verb!`[0-9]'! or \verb!`[\d]'!), and literals (e.g., \verb!`\a'! or \verb!`\x07'!).
Based on an empirical study measuring regex comprehension on 35 pairs of regexes using 180 participants, as well as an empirical study of nearly 14,000 regexes and their features, we identify smelly and non-smelly regex representations. For example, \verb!`aa*'!  is more smelly than  \verb!`a+'!, based on feature usage frequency in source code (conformance to community standards) and understandability. 

%Our results identify preferred representations for four of the five equivalence classes based on mutual agreement between community standards and understandability. For the fifth group on double-bounded repetitions, two recommendations are given depending on the programmer's goals. 
Our contributions are:
\begin{itemize}
\item An approach and study with 180 participants for evaluating regex understandability, 
\item Identification of equivalence classes for regular expressions,
%\item Conducted an empirical study identifying opportunities for regex refactoring  in Python projects based on how regexes are expressed, 
\item {Identification of smelly and non-smelly regex representations to optimize 1) understandability and 2) conformance to community standards, backed by empirical evidence.}
%\item {Identified 3 or so other regex refactorings categories and specific instances that are worthy of further investigations}
%\item {Identified a few regex refactorings that can be eliminated because both options are equally readable}
\end{itemize}

To our knowledge, this is the first work to explore regex comprehension and regex smells. We approach the problem of identifying preferred regex representations by looking at thousands of regexes in Python projects and measuring the understandability of various regex representations using human participants.  %, one using source code artifacts and another using human participants. 
%  (Section~\ref{sec:refactoring}), research questions (Section~\ref{sec:study}), the study of regex representations in Python projects (Section~\ref{communitystudy}), and the regex understandability study using human participants (Section~\ref{sec:understandability}). We discuss the overall analysis results in Section~\ref{sec:rq3}, implications in Section~\ref{sec:discussion},  related work in regexes (Section~\ref{sec:related}), and conclude in Section~\ref{sec:conclusion}.
%\todoLast{can remove for space}

%The rest of this paper is organized as follows:
%
%Related work, study, results, discussion, conclusion.








\section{Research Questions}
\label{sec:study}
%After defining the equivalence classes and potential  regex refactorings as described in Section~\ref{sec:refactoring}, we wanted to know which representations in the equivalence classes  are considered desirable and which might be smelly. Desirability for regexes can be defined many ways, including maintainable,  understandable, and performance. 
%%As prior work has shown that regexes are difficult to read~\cite{}, 
%We focus on refactoring for understandability.

To explore regex comprehension and identify smells, we compare syntactically different regexes that match the same language. That is, they are behaviorally equivalent but expressed differently. This analysis requires the definition of equivalence classes for regexes. By inspecting  nearly 14,000 regexes extracted from Python projects in a publicly available dataset~\cite{chapman2016}, we formed an initial set of five equivalence classes to explore. 

%We define regex understandability two ways. First,  we  present people with regexes exemplifying some of the more common characteristics and ask them comprehension questions along two directions: determine which of a list of strings are matched by the regex, and compose a string that is matched by the regex. Second, assuming that common programming practices are more understandable than uncommon practices, we explore the frequencies of each representation from Figure~\ref{fig:refactoringTree} using thousands of regexes scraped from Python projects. 
Our  research questions are:

\begin{description}
\item[RQ1:] Which regex representations are most smelly based on \emph{understandability} as measured by identifying matching strings and by composing matching strings?
\item[RQ2:] Which regex representations have the strongest \emph{community support} based on how frequently each representation appears in regexes in open source Python projects?
\item[RQ3:] Which regex representations are most desirable (i.e., least smelly) based on both community support and understandability?
\end{description}

Next, we present the equivalence classes, analysis and results for each  question in turn, followed by a unified discussion in Section~\ref{sec:discussion}. 



\begin{figure*}[tb]
\centering
\includegraphics[width=0.8\textwidth]{illustrations/refactoringTree.eps}
\vspace{-6pt}
\caption{Equivalence classes with various representations. DBB = Double-Bounded, SNG = Single Bounded, LWB = Lower Bounded, CCC = Custom Character Class and LIT = Literal. We use concrete regexes in the representations for illustration. However, the A's in the LWB group (or B's in DBB group, S's in SNG group, and so forth) abstractly represent any pattern that could be operated on by a repetition modifier (e.g., literal characters, character classes, or groups). The same is true for the literals used in all the representations. }
\vspace{-6pt}
\label{fig:refactoringTree}
\end{figure*}




%\footnote{same dataset used in prior work~\cite{chapman2016}}
\section{Equivalence Classes}
\label{sec:refactoring}
%After studying over 13,000 distinct regex strings from Python projects, 
We have defined an initial set of equivalence classes for regexes. 
Using the publicly available behavioral clusters from prior work~\cite{}, we manually explored how the various clusters were expressed, and identified several representations that appeared in many of the larger clusters. 
While not a complete set of equivalence classes, this is the first work to explore regex representations and understandability, and these equivalence classes provide an initial testbed that allows us to explore regex comprehension. (Section~\ref{} identifies other equivalence classes left to explore in future work.)

%For example,  \verb!AAA*! and \verb!AA+! are semantically identical, except one uses the star operator (indicating zero or more repetitions) and the other uses the plus operator (indicating one or more repetitions).
%Both match strings with two or more \verb!A!'s.
Figure~\ref{fig:refactoringTree} displays the five equivalence classes in grey boxes and various semantically equivalent \emph{representations} of a regex are shown in white boxes. For example, LWB is an equivalence class with representations that all have a lower bound on repetitions. Regexes \verb!AAA*! and \verb!AA+!  are both members of this class mapping to representations L2 and L3, respectively, along with the L1 representation, \verb!A{2,}!.
%The undirected edges between the representations define possible refactorings.
%Identifying the best direction for each arrow in the possible refactorings is discussed in Section~\ref{sec:rq3}.
%We use concrete regexes in the representations to more clearly illustrate examples of the representations. However, the \verb!A!'s in the LWB group abstractly represent any pattern that could be operated on by a repetition modifier (e.g., literal characters, character classes, or groups). The same is true for the literals used in all the representations. 
Next, we describe each group, the representations, and possible transformations in detail.

\paragraph{Custom Character Class Group}
The Custom Character Class (CCC) group has regex representations that use the custom character class language feature or can be represented by such a feature.
%The character class regex language feature is a fundamental feature found in all language flavors since GREP (check this?).
 A custom character class enables a programmer to specify a set of alternative characters, any of which can match.  For example, the regex \verb!`c[ao]t'! will match both the string ``cat" and the string ``cot" because, between the \verb!c! and \verb!t!, there is a custom character class, \verb![ao]!, that specifies either \verb!a! or \verb!o! (but not both) must be selected.  We use the term \emph{custom} to differentiate these classes  from the default character classes, : \verb!\d!, \verb!\D!, \verb!\w!, \verb!\W!, \verb!\s!, \verb!\S! and \verb!.!,  provided by most regex libraries, though the default classes can be encapsulated in a custom character class, as is the case with the C4 representation.
 % For the purposes of our analysis, a negated custom character class (like \verb![^abc]!) is handled separately.
Next, we provide descriptions of each representation in this equivalence class:

\begin{description}  \itemsep -1pt
\item[C1:] Any pattern that contains a (non-negative) custom character class with  a range feature like \verb![a-f]! as shorthand for all of the characters between `a' and `f' (inclusive) belongs to the C1 node.

\item[C2:] Any pattern that contains a (non-negative) custom character class  without any shorthand representations, specifically ranges or defaults. For example, \verb!`[012]'! is in C2, but \verb!`[0-2]'! is not.

\item[C3:] Any pattern with a character classes expressed using negation, which is indicated by a caret (i.e., \verb!^!) followed by a custom character class specification.
% (including literal characters, default character classes and ranges).  
For example, the pattern \verb![^ao]! matches every character \emph{except} \verb!a! or \verb!o!.  If the applicable character set is known (e.g., ASCII, UTF-8, etc.), then any non-negative character class can be represented as a negative character class.  For example, assuming an ASCII charset that has 128 characters: \verb!\x00-\x7f!, a character class representing the lower half: \verb![\x00-\x3f]! can be represented by negating the upper half: \verb![^\x40-\x7f]!.


\item[C4:] Any pattern using a default character class such as \verb!\d! or \verb!\W! within a (non-negative) character class belongs to the C4 node.  

\item[C5:] While not expressed using a character class, these representations can be transformed into custom character classes by removing the ORs and adding square brackets (e.g., \verb!(\d|a)! in C5 is equivalent to \verb![\da]! in C4). All custom character classes expressed as an OR of length-one sequences, including defaults or other custom classes, are included in C5. Note that because an OR cannot be directly negated, it does not make sense to have an edge between C3 and C5 in Figure~\ref{fig:refactoringTree}, though C3 may be able to transition to C1, C2 or C4 first and then to C5. 
\end{description}

A pattern can belong to multiple representations. For example,  \verb![a-f\d]! belongs to both C1 and C4.  The edge between C1 and C4 represents the opportunity to express the same pattern as \verb![a-f0-9]! by transforming the default digit character class into a range.  This transformed version would only belong to the C1 node.
%\todoNow{add a thing}

\paragraph{Double-Bounded Group}
The Double-Bounded (DBB) group contains all regex patterns that use some repetition defined by a (non-equal) lower and upper boundary.  For example the pattern \verb!pB{1,3}s! represents a \verb!p! followed by one to three sequential \verb!B! patterns, then followed by a single \verb!s!.  This will match ``pBs", ``pBBs", and ``pBBBs".

\begin{description}  \itemsep -1pt
\item[D1:] Any pattern that  uses the curly brace repetition with a lower and upper bound, such as  \verb!pB{1,3}s!, belongs to the D1 node.
Note that  \verb!pB{1,3}s! can become \verb!pBB{0,2}s! by pulling the lower bound out of the curly braces and into the explicit sequence (or visa versa). Nonetheless, it would still be part of D1, though this within-node refactoring on D1 is not discussed in this work.
\item[D2:] Any pattern that uses the questionable (i.e., \verb!?!) modifier implies a lower-bound of zero and an upper-bound of one, and belongs to D2. For example, when a double-bounded regex has zero on the lower bound, as is the case with \verb!pBB{0,2}s!  in D1, transforming it to D2 involves replacing the curly braces with $n$ questionable modifiers, where $n$ is the upper bound,  creating \verb!pBB?B?s!.
\item[D3:] Any pattern that has a repetition with a lower and upper boundary and is expressed using ORs is part of D3.  The example, \verb!pB{1,3}s! would become \verb!pBs|pBBs|pBBs! by expanding on each option in the boundaries.
%Note also that a pattern can belong to multiple nodes in the DBB group, for example, \verb!(a|aa)X?Y{2,4}! belongs to all three nodes.
% =======
% \item[D1:] Any pattern that  uses the curly brace repetition with a lower and upper bound where the upper and lower bounds are different, such as  \verb!pB{1,3}s!, belongs to the D1 node.
% Note that  \verb!pB{1,3}s! can become \verb!pBB{0,2}s! by pulling the lower bound out of the curly braces and into the explicit sequence (or visa versa). Nonetheless, it would still be part of D1, though this within-node refactoring on D1 is not discussed in this work.
% \item[D2:] Any pattern that uses the questionable (i.e., \verb!?!) modifier implies a lower-bound of zero and an upper-bound of one, and belongs to D2. For example, when a double-bounded regex has zero on the lower bound, as is the case with \verb!pBB{0,2}s!  in D1, transforming it to D2 involves replacing the curly braces with $n$ questionable modifiers, where $n$ is the upper bound,  creating \verb!pBB?B?s!.
% \item[D3:] Any pattern that has a repetition with a lower and upper boundary and is expressed using ORs is part of D3.  The example, \verb!pB{1,3}s! would become \verb!pBs|pBBs|pBBS! by expanding on each option in the boundaries. The challenge with identifying membership in this node is recognizing the opportunity to replace the ORs with double-boundaries, which we discuss in Section~\ref{}.
% >>>>>>> 741a48d7abdf9c0f0b7741ca9a47fda9903c3a0f

%\todoNow{make sure to differentiate this clearly from C5}
\end{description}

A pattern can belong to multiple nodes in the DBB group, for example, \verb!(a|aa)X?Y{2,4}! belongs to all three nodes: \verb!Y{2,4}! maps it to D1, \verb!X?!  maps it to D2, and \verb!(a|aa)!  maps it to D3.

% The same functional pattern can be represented as \verb!lol(ol)?(ol)?!, because the questionable (QST) modifier is used.  Note how in general, this procedure is simply pulling out N QST groups from a curly brace style repetition with a zero lower bound and an upper bound of N.  One question mark is equivalent to the curly brace style with a lower bound of 0, and upper bound of 1, so \verb!X?! is equivalent to \verb!X{0,1}!, so we can express \verb!X{0,2}! as \verb!X?X?!.  Any regex using the QST modifier belongs to the D2 node.



\paragraph{Literal Group}
In the Literal (LIT) group, all patterns that are not purely default character classes must use  literal tokens to specify  characters to match.  In  most  languages that support regex libraries, the programmer is able to specify literal tokens in a variety of ways.  Here, we use the ASCII charset in which all characters can be expressed using hex and octal codes such as \verb!\xF1! and \verb!\0108!, respectively.  
%This group defines transformations among various representations of literals.

%Although not all characters can be expressed directly using literal characters typed on the keyboard, the overwhelming majority of patterns do not belong to nodes T2, T3 or T4 because they do not use any of those special features, and so these nodes


\begin{description}  \itemsep -1pt
\item[T1:] Patterns that do not use any hex, wrapped, or octal characters, but use at least one literal character. Special characters are escaped using backslash. 
\item[T2:] Any pattern using a hex token, such as \verb!\x07!.
\item[T3:]  Any pattern with a literal wrapped in square brackets. 
Literal character can be wrapped in brackets to form a custom character classes of size one, such as \verb![x]!. This style is used most often to avoid using a backslash for a special character in the regex language, for example, \verb![{]! which must otherwise be escaped like \verb!\{!.

\item[T4:] Any pattern using an octal token, such as \verb!\007!.
\end{description}

Patterns often fall in multiple of these representations, for example, \verb!abc\007! includes literals \verb!a!, \verb!b!, and \verb!c!, and also octal \verb!\007!, thus belonging to T1 and T4. We also note that 
not all transformations are possible between representations in this group, for example, if a hex representation is used to represent a character that does not appear on the keyboard, a transformation to T1 or T3 will not be possible. 

\paragraph{Lower-Bounded Group}
The Lower-Bounded (LWB) group contains patterns that specify only a lower boundary on  repetitions. This can be expressed using curly braces with a comma after the lower bound but no upper bound, for example \verb!A{2,}! which will match ``AA", ``AAA", ``AAAA", and any number of A's greater or equal to 2.  In Figure~\ref{fig:refactoringTree}, we chose the lower bound repetition threshold of  2 for illustration; in practice this could be any number, including zero.


\begin{description}  \itemsep -1pt
\item[L1:] Any pattern using this curly braces-style lower-bounded repetition belongs to node L1.
\item[L2:] Any pattern using the kleene star, which  means zero-or-more repetitions. %For example, \verb!X*! is equivalent to \verb!X{0,}!.  
\item[L3:] Any pattern using the additional repetition, for example \verb!T+! which means one or more \verb!T!'s.  
%This is equivalent to \verb!T{1,}!.  
\end{description}

Patterns often fall into multiple nodes in this equivalence class. For example, with \verb!A+B*!,  \verb!A+! maps it to L3 and \verb!B*! maps it to L2. Note that refactoring from L1 to L3 and L2 to L3 is not always possible when the lower bound is zero and the pattern is not repeated in sequence (e.g., \verb!`A*'! or \verb!`A{0,}'!).

\paragraph{Single-Bounded Group} 
The Single-Bounded (SNG) equivalence class contains  three representations in which each regex has a fixed number of repetitions of some element. The important factor distinguishing this group from DBB and LWB is that there is a single finite number of repetitions, rather than a bounded range on the number of repetitions (DBB) or a lower bound on the number of repetitions (LWB).


\begin{description}  \itemsep -1pt
\item[S1:] Any pattern with a single repetition boundary in curly braces belongs to S1. For example,   \verb!S{3}!, states that S appears exactly three times in sequence.
\item[S2:] Any pattern that is explicitly repeated two or more times and could use repetition operators is part of S2.
\item[S3:] Any pattern with a double-bound in which the upper and lower bounds are same belong to S3. For example, \verb!S{3,3}! states \verb!S! appears a minimum of 3 and maximum of 3 times.
\end{description}

The pattern \verb!fa[lmnop][lmnop][lmnop]! is a member of S2 as \verb![lmnop]! is repeated three times, and it could be transformed to \verb!fa[lmnop]{3}! in S1 or \verb!fa[lmnop]{3,3}! in S3.

%\paragraph{Example}
%Regular expressions will often belong to multiple representations in multiple equivalence classes described.
%Using an example from a Python project from our analysis, the regex \verb!`[^ ]*\.[A-Z]{3}'! is a member of S1, L2, C1, C3, and T1. This is because \verb!`[^ ]'! maps it to C3, \verb!`[^ ]*'! maps it to L2, \verb!`[A-Z]'! maps it to C1, \verb!`\.'! maps it to T1, and \verb!`[A-Z]{3}'! maps it to S1.
%As examples of refactorings, moving from S1 to S2 would be possible by replacing  \verb!`[A-Z]{3}'! with  \verb!`[A-Z][A-Z][A-Z]'! and moving from L2 to L1 would replace \verb!`[^ ]*'! with \verb!`[^ ]{0,}'!, resulting in a refactored regex of:  \verb!`[^ ]{0,}\.[A-Z][A-Z][A-Z]'!.

% and could be refactored to \emph{S3} as \verb!`[^ ]*\.[A-Z]{3,3}'!  or to \emph{S2} as \verb!`[^ ]*\.[A-Z][A-Z][A-Z]'!, depending on programmer preferences.
%\todoNow{can we have examples from Python projects for all the groups???}





%First we define a 'Functional Regex'(FR) as some regex that performs in a specific way.  For many FRs, there are several concrete ways to express a single FR.
%We define a concrete regex(CR) as a regex expressed with a particular pattern String.
%Here is one illustration of these definitions:
%
%\todoNow{create some examples for these terms}
%
%We identified 10 loose groups of FRs, described in this table:
%
%\todoNow{create a table explaining the 10 groups}
%
%For each of these groups we created either two concrete versions of three FRs or three concrete versions of two FRs.
%
%Each of the 10 categories had 6 concrete versions of some FR and so there are 60 CRs.  For each CR, we selected 5 \emph{example strings} designed to test the understanding of the CR.  The idea is that different CRs may have different levels of readability, even when they are representing the same FR.  We define readability as the ability to look at the CR and determine if an \emph{example string} can be matched by it or not.
%
%\todoNow{create some illustration of one matching subtask}
%

\begin{table*}[ht]
%\begin{small}
\begin{center}
\caption{How frequently is each alternative expression style used?}
\label{table:nodeCount}
\begin{tabular}
{lll@{}rrrr}
Node & Description & Example & nPatterns & \% patterns & nProjects & \% projects \\ 
\toprule[0.16em]
%corpusProjectIDs.size(): 1544
C1 & char class using ranges & \begin{minipage}{1.5in}\begin{verbatim}
'^[1-9][0-9]*$'\end{verbatim}\end{minipage}
 & 2,479 & 18.2\% & 810 & 52.5\%\\
C2 & char class explicitly listing all chars & \begin{minipage}{1.5in}\begin{verbatim}
'[aeiouy]'\end{verbatim}\end{minipage}
 & 1,903 & 14.0\% & 715 & 46.3\%\\
C3 & any negated char class & \begin{minipage}{1.5in}\begin{verbatim}
'[^A-Za-z0-9.]+'\end{verbatim}\end{minipage}
 & 1,935 & 14.2\% & 776 & 50.3\%\\
C4 & char class using defaults & \begin{minipage}{1.5in}\begin{verbatim}
'[-+\d.]'\end{verbatim}\end{minipage}
 & 840 & 6.2\% & 414 & 26.8\%\\
C5 & an OR of length-one sub-patterns & \begin{minipage}{1.5in}\begin{verbatim}
'(@|<|>|-|!)'\end{verbatim}\end{minipage}
 & 245 & 1.8\% & 239 & 15.5\%\\
\midrule
D1 & curly brace repetition like \{M,N\} with M<N & \begin{minipage}{1.5in}\begin{verbatim}
'^x{1,4}$'\end{verbatim}\end{minipage}
 & 346 & 2.5\% & 234 & 15.2\%\\
D2 & zero-or-one repetition using question mark & \begin{minipage}{1.5in}\begin{verbatim}
'^http(s)?://'\end{verbatim}\end{minipage}
 & 1,871 & 13.8\% & 646 & 41.8\%\\
D3 & repetition expressed using an OR & \begin{minipage}{1.5in}\begin{verbatim}
'^(Q|QQ)\<(.+)\>$'\end{verbatim}\end{minipage}
 & 10 & .1\% & 27 & 1.7\%\\
\midrule
T1 & no HEX, OCT or char-class-wrapped literals & \begin{minipage}{1.5in}\begin{verbatim}
'get_tag'\end{verbatim}\end{minipage}
 & 12,482 & 91.8\% & 1,485 & 96.2\%\\
T2 & has HEX literal like \verb!\xF5! & \begin{minipage}{1.5in}\begin{verbatim}
'[\x80-\xff]'\end{verbatim}\end{minipage}
 & 479 & 3.5\% & 243 & 15.7\%\\
T3 & has char-class-wrapped literals like [\$] & \begin{minipage}{1.5in}\begin{verbatim}
'[$][{]\d+:([^}]+)[}]'\end{verbatim}\end{minipage}
 & 307 & 2.3\% & 268 & 17.4\%\\
T4 & has OCT literal like \verb!\0177! & \begin{minipage}{1.5in}\begin{verbatim}
'[\041-\176]+:$'\end{verbatim}\end{minipage}
 & 14 & .1\% & 37 & 2.4\%\\
\midrule
L1 & curly brace repetition like \{M,\} & \begin{minipage}{1.5in}\begin{verbatim}
'(DN)[0-9]{4,}'\end{verbatim}\end{minipage}
 & 91 & .7\% & 166 & 10.8\%\\
L2 & zero-or-more repetition using kleene star & \begin{minipage}{1.5in}\begin{verbatim}
'\s*(#.*)?$'\end{verbatim}\end{minipage}
 & 6,017 & 44.3\% & 1,097 & 71.0\%\\
L3 & one-or-more repetition using plus & \begin{minipage}{1.5in}\begin{verbatim}
'[A-Z][a-z]+'\end{verbatim}\end{minipage}
 & 6,003 & 44.1\% & 1,207 & 78.2\%\\
\midrule
S1 & curly brace repetition like \{M\} & \begin{minipage}{1.5in}\begin{verbatim}
'^[a-f0-9]{40}$'\end{verbatim}\end{minipage}
 & 581 & 4.3\% & 340 & 22.0\%\\
S2 & explicit sequential repetition & \begin{minipage}{1.5in}\begin{verbatim}
'ff:ff:ff:ff:ff:ff'\end{verbatim}\end{minipage}
 & 3,378 & 24.8\% & 861 & 55.8\%\\
S3 & curly brace repetition like \{M,M\} & \begin{minipage}{1.5in}\begin{verbatim}
'U[\dA-F]{5,5}'\end{verbatim}\end{minipage}
 & 27 & .2\% & 32 & 2.1\%\\
\bottomrule[0.13em]
\end{tabular}
\end{center}
%\end{small}
\vspace{-6pt}
\vspace{-6pt}
\end{table*}




\section{Understandability Study (RQ1)}
\label{sec:understandability}
This study presents  programmers with regexes and asks comprehension questions. By comparing the understandability of semantically equivalent regexes that match the same language but have different syntactic representations, we aim to identify understandability code smells.
This study was  implemented on Amazon's Mechanical Turk with 180 participants. A total of 35 pairs of regex were evaluated. Each regex pattern was evaluated by 30 participants.
%The patterns used were designed to belong to various representations in Figure~\ref{fig:refactoringTree}.







\subsection{Metrics}
\label{sec:understadningmetric}
 We measure the understandability of regexes using two complementary metrics, \emph{matching} and \emph{compostition}.


\textbf{Matching:}
 Given a pattern and a set of strings, a participant determines by inspection which strings will be matched by the pattern. There are four possible responses for each string, \emph{matches}, \emph{not a match}, \emph{unsure}, or blank. An example from our study is shown in Figure~\ref{fig:exampleQuestion}.

 The percentage of correct responses, disregarding blanks and unsure responses, is the matching score.
 For example, consider regex pattern \verb!`RR*'! and five strings shown in Table~\ref{matchingmetric}, and the responses from four participants in the \emph{P1}, \emph{P2}, \emph{P3} and \emph{P4} columns.
 The oracle has the first three strings matching since they each contain at least one \verb!R! character. \emph{P1} answers correctly for the first three strings but incorrectly on the fourth string, so the matching score is $4/5 = 0.80$. \emph{P2} incorrectly thinks that the second string is not a match, so the score is also $4/5 = 0.80$.  \emph{P3} marks `unsure' for the third string and so the total number of attempted matching questions is 4. \emph{P3} is incorrect about the second and fourth string, so they score $2/4 = 0.50$.  For \emph{P4}, we only have data for the first and second strings, since the other three are blank.  \emph{P4} marks `unsure' for the second string so only one matching question has been attempted;  the matching score is $1/1 = 1.00$.

Blanks were incorporated into the metric because questions were occasionally left blank in the study. Unsure responses were provided as an option so not to bias the  results through blind guessing. These situations did not occur very frequently. 
%Only 1.1\% of the responses were left blank and only 3.8\% of the responses were marked as unsure.  %We refer to a response with all blank or unsure responses as an `NA'. 
Out of 1800 questions (180 participants * 10 questions each), only 1.8\%(32) were impacted by a blank or unsure response (never more than four out of 30 per pattern).


\begin{figure}[tb]
\centering
\includegraphics[width=0.75\columnwidth]{illustrations/ExampleQuestion}
\vspace{-12pt}
\caption{Example of one HIT Question}
\vspace{-6pt}
\label{fig:exampleQuestion}
\end{figure}



\begin{table} [t]
\caption{Matching metric example \label{matchingmetric}}
\begin{center}
%\begin{small}
\begin{tabular} {|cl | c c c c c|} \hline
\textbf{String} & \verb!`RR*'! & \textbf{Oracle} & \textbf{P1} & \textbf{P2} & \textbf{P3}& \textbf{P4}\\ \hline
1 & ``ARROW"    & \checkmark    & \checkmark    & \checkmark    & \checkmark    & \checkmark \\
2 & ``qRs"      & \checkmark    & \checkmark    & \xmark        & \xmark        & ?\\
3 & ``R0R"      & \checkmark    & \checkmark    & \checkmark    & ?             & -\\
4 & ``qrs"      & \xmark        & \checkmark    & \xmark        & \checkmark    & -\\
5 & ``98"       & \xmark        & \xmark        & \xmark        & \xmark        & -\\
\hline
  & Score       & 1.00          & 0.80          & 0.80          & 0.50          & 1.00\\ \hline
\multicolumn{7}{l}{}\\ 
\multicolumn{7}{l}{\checkmark = match, \xmark = not a match, ? = unsure, -- = left blank}\\
\end{tabular}
%\end{small}
\end{center}
\vspace{-6pt}
\vspace{-6pt}
\end{table}



\textbf{Composition:}
Given a pattern, a participant composes a string they think it matches. If the participant is accurate, a composition score of 1 is assigned, otherwise 0.  For example, given the pattern \verb!`(q4fab|ab)'! from our study, the string, ``xyzq4fab" matches  and gets a score of 1, but the string, ``acb" does not match and gets  a score of 0.

To determine a match, each pattern was compiled using the \emph{java.util.regex} library. A \emph{java.util.regex.Matcher} \verb!m! object was created for each composed string using the compiled pattern.  If \verb!m.find()! returned true, then that composed string was a match and scored 1, otherwise it scored 0.


%\begin{table*}\begin{small}\begin{center}\caption{Averaged Info About Edges (sorted by lowest of either p-value)}\label{table:testedEdgesTable}\begin{tabular}
%{llccccccc}
%Index & Representations & Pairs & Match1 & Match2 & $H_0: \mu_{match1} = \mu_{match2}$ & Compose1 & Compose2 &  $H_0: \mu_{comp1} = \mu_{comp2}$ \\
%\toprule[0.16em]
%E1 & T1 -- T4 & 2 & 0.80 & 0.60 & 0.001 & 0.87 & 0.37 & \textbf{<0.001}\\
%E2 & D2 -- D3 & 2 & 0.78 & 0.87 & \textbf{0.011} & 0.88 & 0.97 & 0.085\\
%E3 & L2 -- L3 & 3 & 0.86 & 0.91 & \textbf{0.032} & 0.91 & 0.98 & 0.052\\
%\midrule[0.16em]
%E4 & C2 -- C5 & 4 & 0.85 & 0.86 & 0.602 & 0.88 & 0.95 & {0.063}\\
%E5 & C2 -- C4 & 1 & 0.83 & 0.92 & {0.075} & 0.60 & 0.67 & 0.601\\
%
%E6 & D1 -- D2 & 2 & 0.84 & 0.78 & 0.120 & 0.93 & 0.88 & 0.347\\
%E7 & C1 -- C2 & 2 & 0.94 & 0.90 & 0.121 & 0.93 & 0.90 & 0.514\\
%E8 & T2 -- T4 & 2 & 0.84 & 0.81 & 0.498 & 0.65 & 0.52 & 0.141\\
%E9 & C1 -- C5 & 2 & 0.94 & 0.90 & 0.287 & 0.93 & 0.93 & 1.000\\
%E10 & T1 -- T3 & 3 & 0.88 & 0.86 & 0.320 & 0.72 & 0.76 & 0.613\\
%E11 & D1 -- D3 & 2 & 0.84 & 0.87 & 0.349 & 0.93 & 0.97 & 0.408\\
%E12 & C1 -- C4 & 6 & 0.87 & 0.84 & 0.352 & 0.86 & 0.83 & 0.465\\
%E13 & C3 -- C4 & 2 & 0.61 & 0.67 & 0.593 & 0.75 & 0.82 & 0.379\\
%E14 & S1 -- S2 & 3 & 0.85 & 0.86 & 0.776 & 0.88 & 0.90 & 0.638\\
%\bottomrule[0.13em]\end{tabular}\end{center}\end{small}
%\vspace{-6pt}
%\vspace{-6pt}
%\end{table*}


\begin{table*}\begin{footnotesize}\begin{center}\caption{Matching and composition scores for edges from Figure~\ref{fig:refactoringTree} evaluated by 180 human study participants. Preferred nodes are bolded, and examples of regexes from the study that reside in the preferred nodes are provided. The number of pairs indicates how many semantic pairs for the edge were evaluated. The horizontal line indicates which comparisons are statistically different with $\alpha = 0.10$. The table is sorted by lowest of either p-value.}\label{table:testedEdgesTable}\begin{tabular}
{llclcccccc}
\textbf{Index} & \textbf{Nodes} & \textbf{Pairs} & \textbf{Example Preferred Regex} & \textbf{Match1} & \textbf{Match2} & \textbf{$H_0^{match} $} & \textbf{Compose1} & \textbf{Compose2} &  \textbf{$H_0^{comp}$} \\
\toprule[0.16em]
E1 & \textbf{T1} -- T4 & 2 & \verb!`t[:;]+p'! & 80\% & 60\% & 0.001 & 87\% & 37\% & $<$\textbf{0.001}\\
E2 & D2 -- \textbf{D3} & 2 &\verb!`(q4fab|ab)'! & 78\% & 87\% & \textbf{0.011} & 88\% & 97\% & 0.085\\
E3 & C2 -- \textbf{C5} & 4 & \verb!`tri(a|b|c|d|e|f)3'!& 85\% & 86\% & 0.602 & 88\% & 95\% & \textbf{0.063}\\
E4 & C2 -- \textbf{C4} & 1 & \verb!`[\s]'! &83\% & 92\% & \textbf{0.075} & 60\% & 67\% & 0.601\\
\midrule[0.05em]
E5 & L2 -- \textbf{L3} & 2 & \verb!`R+'!& 86\% & 91\% & 0.118 & 97\% & 100\% & 0.159\\
E6 & \textbf{D1} -- D2 & 2 & \verb!`(dee(do){1,2})'!&84\% & 78\% & 0.120 & 93\% & 88\% & 0.347\\
E7 & \textbf{C1} -- C2 & 2 & \verb!`tri[a-f]3'! &94\% & 90\% & 0.121 & 93\% & 90\% & 0.514\\
E8 & \textbf{T2} -- T4 & 2 & \verb!`xyz[\x5b-\x5f]'! &84\% & 81\% & 0.498 & 65\% & 52\% & 0.141\\
E9 & \textbf{C1} -- C5 & 2 & \verb!`no[w-z]5'! &94\% & 90\% & 0.287 & 93\% & 93\% & 1.000\\
E10 & \textbf{T1} -- \textbf{T3} & 3 & \verb!`t\.\$+\d+\*'! \textbf{--} \verb!`t[.][$]+\d+[*]'!&88\% & 86\% & 0.320 & 72\% & 76\% & 0.613\\
E11 & D1 -- \textbf{D3} & 2 & \verb!`(deedo|deedodo)'!&84\% & 87\% & 0.349 & 93\% & 97\% & 0.408\\
E12 & \textbf{C1} -- C4 & 6 & \verb!`&([A-Za-z0-9_]+);'! &87\% & 84\% & 0.352 & 86\% & 83\% & 0.465\\
E13 & C3 -- \textbf{C4} & 2 & \verb!`[\D]'! &61\% & 67\% & 0.593 & 75\% & 82\% & 0.379\\
E14 & S1 -- \textbf{S2} & 3 & \verb!`%([0-9a-fA-F][0-9a-fA-F])'! &85\% & 86\% & 0.776 & 88\% & 90\% & 0.638\\
\bottomrule[0.13em]\end{tabular}\end{center}\end{footnotesize}\end{table*}



\subsection{Design}
%\todoNow{needs to be updated with respect to no C1,T1 nodes}
This study was implemented on the Amazon's Mechanical Turk (MTurk),  a crowdsourcing platform in which requesters can create human intelligence tasks (HITs) for completion by workers. 
%Each HIT is designed to be completed in a fixed amount of time and workers are compensated with money if their work is satisfactory. Requesters can screen workers by requiring each to complete a qualification test prior to completing any HITs.

\subsubsection{Worker Qualification}
Workers qualified to participate  by answering questions regarding some basics of regex knowledge. These questions were multiple-choice and asked the worker to analyze the following patterns: \verb!`a+'!, \verb!`(r|z)'!, \verb!`\d'!, \verb!`q*'!, and \verb!`[p-s]'!. To pass the qualification test, workers had to answer four of the five questions correctly.

\subsubsection{Tasks}
Using the patterns in the corpus as a guide, we created 60 regex patterns that were grouped into 26 semantic equivalence groups. 
 There were 18 groups with two regexes that target various edges in the equivalence classes. 
The other eight semantic groups had three regexes each, forming 42 total pairs. 
These semantic groups were intended to explore edges in the equivalence classes. In this way, we can draw conclusions by comparing between representations since the regexes evaluated were semantically equivalent. 

To form the semantic groups, we took a regex from the corpus, matched it to a representation in Figure~\ref{fig:refactoringTree}, trimmed it down so it contained little more than just the feature of interest, and then created other regexes that are semantically equivalent but belong to other nodes in the equivalence class. For example, a semantic group with regexes \verb!`((q4f){0,1}ab)'!, \verb!`((q4f)?ab)'!, and \verb!`(q4fab|ab)'! belong to representations D1, D2, and D3, respectively. 
A  group with regexes \verb!`([0-9]+)\.([0-9]+)'! and  \verb!`(\d+)\.(\d+)'! is intended to evaluate the edge between C1 and C4.
We note that if we only used regexes from the corpus, we would have had regexes with different semantics at each node, or with additional language features, making comparisons in comprehension difficult. 




%Using the patterns in the corpus as a guide, we created six metagroups containing three pairs of patterns focusing on:
%\begin{itemize}
%\item S1 vs S2
%\item the digit default character class vs C1
%\item the word default character class vs C1
%\item negated digits and words vs C3, whitespace vs C2
%\item additional vs kleene repetition
%\item wrapping vs escaping literal characters
%\end{itemize}
%and four metagroups containing two triplets of patterns focusing on
%\begin{itemize}
%\item octal vs hex vs literal
%\item D1 vs D2 vs D3
%\item C1 vs C2 vs C5
%\item octal vs literal and C2 vs C5
%\end{itemize}
%
%Each of these 10 metagroups contains 6 strings, resulting in a total of 60 regex patterns.  These patterns are logically partitioned into 26 semantic equivalence groups (18 from pairs, 8 from triples). 

For each of the 26 semantic groups, we created five strings for the study, where at least one matched and at least one did not match. These were used to compute the matching metric.

Once all the patterns and matching strings were collected, we created tasks for the MTurk participants as follows:
randomly select a pattern from each of the 26 semantic groups. Randomize the order of these 10 patterns, as well as the order of the matching strings for each pattern. After adding a question asking the participant to compose a string that each pattern matches, this creates one task on MTurk, such as the example in Figure~\ref{fig:exampleQuestion}.   This process was completed until each of the 60 regexes appeared in 30 HITs, resulting in a total of 180 total unique HITs.
%An example of a single regex pattern, the five matching strings and the space for composing a string is shown in 


\subsubsection{Implementation}
Workers were paid \$3.00 for successfully completing a HIT, and were only allowed to complete  one HIT.  The average completion time for accepted HITs was 682 seconds (11 mins, 22 secs).
%A total of 241 HITs were submitted - of those 55 were rejected.
%, and 6 duplicates were ignored, always using the first accepted submission so as to obtain a value for each of the 180 distinct tasks.
A total of 55 HITs were rejected, and  of those, 48 were rushed through by leaving many answers blank, four were rejected because a worker had submitted more than one HIT, one was rejected for not answering composition sections, and one was rejected because it was missing data for 3 questions.  Rejected HITs were returned to MTurk to be completed by others.


%
%
%
%\begin{figure}[tp]
%\begin{small}
%\fbox{\parbox{\columnwidth}{
%\begin{enumerate}
%\item
%\begin{tabular} {lrr}
%\textbf{What is your gender?} & \textbf{n} & \textbf{\%}\\ \hline
%Male & 149 & 83\%\\
%Female & 27& 15\%\\
%Prefer not to say & 4& 2\%
%\end{tabular}
%\item \textbf{What is your age?} \\
%$\mu = 31$, $\sigma = 9.3$
%
%\item
%
%\begin{tabular} {l |rr}
%\textbf{Education Level?} & \textbf{n} & \textbf{\%}\\ \hline
%High School & 5 & 3\%\\
%Some college, no degree & 46 & 26\%\\
% Associates degree & 14 & 8\%\\
%Bachelors degree & 78 & 43\%\\
%Graduate degree & 37 & 21\%\\
%\end{tabular}
%\item
%\begin{tabular} {lrr}
%\textbf{Familiarity with regexes?} & \textbf{n} & \textbf{\%}\\ \hline
%Not familiar at all & 5 & 3\%\\
%Somewhat not familiar & 16 & 9\%\\
%Not sure & 2 & 1\%\\
%Somewhat familiar & 121 & 67\%\\
%Very familiar & 36 & 20\%\\
%\end{tabular}
%\item \textbf{How many regexes do you compose each year?} \\
%$\mu = 67$, $\sigma = 173$
%\item \textbf{How many regexes (not written by you) do you read each year?} \\
%$\mu = 116$, $\sigma = 275$
%%\item In what contexts do you use regexes? \\
%\end{enumerate}
%}}
%\caption{Participant Profiles, $n=180$ \todoLast{can remove this for space} \label{participantprofile}}
%\end{small}
%\end{figure}
%


\subsubsection{Participants}

In total, there were 180 participants.
A majority were male (83\%) with an average age of 31. Most had
at least an Associates degree (72\%), were at least somewhat familiar with regexes (87\%), and have prior programming experience (84\%). 
%On average,
%participants compose 67 regexes per year with a range from 0 to 1000. 
%Participants read more regexes than they write with an average of 116 and a range from 0 to 2000.
%Figure~\ref{participantprofile} summarizes the self-reported participant characteristics from the qualification survey.


%\todoNow{in study section present choices about pairwise vs random selection for nodes.}


\subsection{Analysis}
For each of the 180 HITs, we computed a matching and composition score for each of the 10 regexes, using the metrics described in Section~\ref{sec:understadningmetric}. This allowed us to compute and then average 26-30 values for each metric  for each of the 60 regexes (fewer than 30 values were used if all the responses in a matching question were a combination of blanks and unsure). %Next, we computed average scores for matching and composition per regex.
%\todoLast{Mentioning NAs here?}

%Each regex was a member of one of 26 groupings of equivalent regexes. 
%These groupings allow pairwise comparisons of the metrics values to determine which representation of the regex was most understandable and the direction of a refactoring for understandability. 
Among all the semantic groups, we performed 42 pairwise comparisons of the matching and composition scores  (i.e., one comparison for each group of size two and three comparisons within each group of size three).
For example, one group had regexes, \verb!RR*! and \verb!R+!, which  represent a transformation between L2 and L3. The former had an average matching of 86\% and the latter had an average matching of 92\%. The average composition score for the former was 97\% and 100\% for the latter. Thus, the community found \verb!R+! from L3 more understandable. 
There was one other pairwise comparisons performed between the L2 and L3 group, using regexes pair \verb!zaa*! and \verb!za+'!. %, and regexes pair \verb!\..*! and \verb!\.+'!. 
Considering both regex pairs, the overall matching average for the regexes belonging to L2 was 0.86 and 0.91 for L3. 
The overall composition score for L2 was 0.97 and 1.00 for L3. Thus, the community found L3 to be more understandable than L2, from the perspective of both understandability metrics, suggesting that L2 is generally smelly, though the results are not significant. 

This information is presented in summary in Table~\ref{table:testedEdgesTable}, with this specific example appearing in the E5 row. 
\emph{Index}  enumerates the edges  evaluated in this experiment, \emph{Nodes} lists the representations, \emph{Pairs} shows the number of comparisons, \emph{Example Preferred Regex} shows a regex from the preferred node (bolded in \emph{Pairs} column), \emph{Match1} and \emph{Match2} give the matching scores for the first and second representations, respectively, and $H_0: \mu_{match1} = \mu_{match2}$ uses the Mann-Whitney test of means to compare the matching scores, and presents the p-values. The last three columns list the average composition scores for the representations and the p-value, also using the Mann-Whitney test of means.

%60 strings
%42 comparisons
%18@2, 8@3
%
%M6R1 ? group 3, 3 comparisons
%- 1 comparisons
%- 0 strings
%
%M3R0 ? group 3, 3 comparisons
%- 1 comparisons
%- 0 strings
%
%M3R1 ? group 3, 3 comparisons
%- 2 comparisons
%- 1 string
%
%M3R0 ? group 3, 3 comparisons
%- 2 comparisons
%- 1 string
%
%58 strings
%36 comparisons


Although we performed 42 pairwise comparisons,  we had to drop seven comparisons  due design flaws. One is that the regexes evaluated, \verb!\..*! and \verb!\.+'!. are not semantically equivalent (the former is missing an escape and should be \verb!\.\.*!).
The other six were dropped since the patterns performed transformations from multiple equivalence classes. For example, pattern \verb!([\072\073])! is in C2 and T4, and was grouped with pattern \verb!(:|;)! in C5, T1, so it was not clear if any differences in understandability were due to the transformation between C2 and C5, or T4 and T1. However, the third member of the group, \verb!([:;])!, could be compared with both, since it is a member of T1 and C2, so comparing it to \verb!([\072\073])! evaluates the transformation between T1 and T4, and comparing to \verb!(:|;)! evaluates the transformation between C2 and C5. The end result is 35 pairwise comparisons across 14 edges from Figure~\ref{fig:refactoringTree}.

\subsection{Results}
Table~\ref{table:testedEdgesTable} presents the results of the understandability analysis. A horizontal line separates the first three edges from the bottom 11. In E1 through E4, there is a statistically significant difference between the representations for at least one of the metrics considering $\alpha = 0.10$.  These represent the strongest evidence for code smells, suggesting that T4, D2, and C2 are less understandable. 

We note again that participants were able to select \emph{unsure} when they were not sure if a string would be matched by a pattern (Figure~\ref{fig:exampleQuestion}). From a comprehension perspective, this indicates some level of confusion and is worth exploring.
% and we can use that to further corroborate the understandability analysis.

%\begin{table*}
%\centering
%\caption{Average Unsure Responses Per Pattern By Node (fewer unsures on the left)}\label{table:unsureResults}
%\begin{tabular}{|| l || cccc || cccc || || cccc || cccc ||}
%                & \multicolumn{4}{c||}{>=Q0(0.67)}         & \multicolumn{4}{c||||}{>=Q1(1.25)}   & \multicolumn{4}{c||}{>=Q2(1.94)}    &  \multicolumn{4}{c||}{>=Q3(2.54)}  \\ \hline
%Node     & L3 & D3 & C2 & C1 & L2 & S2 & S1 & C4 & D1 & C5 & C3 & D2 & T1 & T3 & T2 & T4 \\
%% Number of Patterns - reversed & 4 & 2 & 3 & 3 & 2 & 2 & 4 & 2 & 9 & 3 & 3 & 3 & 8 & 5 & 2 & 3\\
%Unsure Responses Per Pattern & 0.7 & 1 & 1 & 1 & 1.3 & 1.7 & 1.7 & 1.9 & 2 & 2 & 2 & 2.5 & 2.7 & 2.7 & 5.5 & 8.5\\
%\end{tabular}
%\end{table*}
For each pattern, we counted the number of responses containing at least one unsure, representing confusion.
We then grouped the patterns into their representation nodes and computed an average of unsures per pattern.
A higher number may indicate difficulty in comprehending a pattern from that node.
Overall, the highest number of unsure responses came from T4 and T2, which present octal and hex representations of characters. The least number of unsure responses were in L3 and D3.
%, which are both shown to be understandable by looking at E2 and E3 in Table~\ref{table:testedEdgesTable}.
%These nodes and their average number of unsure responses are organized by quartile in Table~\ref{table:unsureResults}.
These results mirror the understandability analysis, which indicates that L3 and D3 are preferred. 
% for the LIT group (i.e., $\overrightarrow{T4 T1}$), the DBB group (i.e.,  $\overrightarrow{D2 D3}$), and the LWB group (i.e., $\overrightarrow{L2 L3}$) because the more understandable node has the least unsures of its group.
% The findings for D3 and D2 are contradictory, however, as  and further study is needed.
%  and the number of unsures may be too small to indicate anything, except for T2 and T4.  The one pattern from T4 that had the most unsures of any pattern (i.e., 10 out of 30) was \verb!`xyz[\0133-\0140]'!, so this may have been the least understandable pattern that we tested.




\section{Community Support Study (RQ2)}
\label{communitystudy}
The goal of this study is to understand how frequently each of the regex representations appears in source code. Based on the results, we identify preferred representations using popularity in source code.



\subsection{Artifacts}
To determine how common each regex representations is, we collected and analyzed 
regexes from GitHub projects. 
We  targeted Python as it is a popular programming language with a strong presence on GitHub, being the fourth most common language after Java, Javascript and Ruby. Further, Python's regex pattern language is close enough to other regex libraries that our conclusions are likely to generalize.

We collected and analyzed static invocations to the Python {\tt re} library.
Figure~\ref{fig:exampleUsage} presents an example  with key components labeled. The \emph{function} called is {\tt re.compile}.
The \emph{pattern} defines what strings will be matched and the \emph{flag} {\tt re.MULTILINE} modifies the rules used by the regex engine when matching.
When executed, a regex object {\tt r1} is created
%The regex pattern is an ordered series of regular expression language feature tokens.
and it will  match if it finds a zero at the end of a line, or a (possibly negative) integer at the end of a line (i.e., due to the {\tt -?} sequence denoting zero or one instance of the {\tt -}).

\begin{figure}[tb]
\centering
\includegraphics[width=\columnwidth]{illustrations/exampleUsage.eps}
\vspace{-12pt}
\caption{Example of one regex library invocation}
\vspace{-6pt}
\label{fig:exampleUsage}
\end{figure}

Our goal was to collect regex patterns from a variety of projects to represent the breadth of how developers use regexes.
We scraped 3,898 projects containing Python code using the GitHub API. This was done by systematically selecting repository IDs, checking the repository for Python files, and retaining the project if Python was found. After dividing eight million repository IDs into 32 groups, we scanned from the beginning until we had collected approximately four thousand Python projects.
%We did so  by dividing a range of about 8 million repo IDs
%into 32 sections of equal size and scanning  for Python projects from the beginning of those
%segments until we ran out of memory.
At that point, we felt we had enough data
to do an analysis without further perfecting our mining techniques.

To identify invocations of the {\tt re} module, we built
the AST of each Python file in each project. In most projects, almost all {\tt re} invocations are present in the
most recent version of a project, but to be more thorough, we also scanned up
to 19 earlier versions.
%The number 20 was chosen to try and maximize returns on
%computing resources invested after observing the scanning process in many hours
%of trial scans.
% If the project had fewer than 20 commits, then all commits were scanned.
% The most recent commit was always included, and the spacing between all other chosen commits was determined by dividing the remaining number of commits by 19 (rounding as needed).
All regex patterns were retained, sans duplicates.
%Within a project, a duplicate utilization was marked when two versions of the same file have the same function, pattern and flags.
In the end, we observed and recorded 16,088 non-duplicate patterns in 1,645 projects.
%\todoLast{1544 may be a white lie...the 13K+ patterns come from 1544 projects, but the 54k utilizations (before pruning) probably come from something like 1900 projects, and that number is somewhere in the git history of tour de source}

In collecting the set of distinct patterns for analysis,  we ignore the 12.7\%  of {\tt re} invocations using flags, which can alter regex behavior.  An additional 6.5\% of {\tt re} invocations contained patterns that could not be compiled because the pattern was non-static (e.g., used some runtime variable).
%The remaining 80.8\% (43,525) of the utilizations were collapsed into 13,711 distinct pattern strings.
This parser was unable to support 0.8\% (114) due to error.
% of the patterns due to unsupported unicode characters.  Another 0.2\% (25) of the patterns used regex features that we  chose to exclude because they appeared very rarely (e.g., reference conditions).  An additional 0.1\% (16) of the patterns were excluded because they were empty or otherwise malformed so as to cause a parsing error.
After removing all problematic patterns as described and collapsing on duplicates, we ended up with 13,597 distinct patterns from 1,544 projects.



\subsection{Metrics}
\label{sec:communitymetric}
We measure community support by matching each regex in the corpus to the representations in Figure~\ref{fig:refactoringTree} and counting the number of \emph{patterns} that contain the representation and the number of \emph{projects} that contain the representation.
%A \emph{pattern} is extracted from a utilization, as shown in Figure~\ref{fig:exampleUsage}.
Note that a regex can belong to multiple representations, and a regex can belong to multiple projects since we collapsed duplicates. % and only analyze the distinct regex patterns.
% that represent 43,525 regex utilizations across the projects.\todoMid{feels weird to hear this again right away, maybe simplify the metrics paragraph}
%For this frequency analysis, we focus on patterns and the number of projects the patterns appear in.
%To determine how often each representation appears in the wild, we extract regex patterns from source code and measure if a representation matches (part of) the pattern.
%
%
%\paragraph{Patterns}



%\paragraph{Projects}

%The process for deciding if a particular pattern belongs to a particular node is described in detail in Section~\ref{communityanalysis}.





\subsection{Analysis}
\label{communityanalysis}
To determine how many of the representations match patterns in the corpus, we performed an analysis using the PCRE parser and by representing the regexes as token streams, depending on the characteristics of the representation. Our analysis code is available on GitHub\footnote{\url{https://github.com/softwarekitty/regex_readability_study}}. Next, we describe the process in detail:

\subsubsection{Presence of a Feature}
For the representations that only require a particular feature to be present, such as the question-mark in D2, the features identified by the PCRE parser were used to decide membership of patterns in nodes.
These feature-requiring nodes are as follows: D1 requires double-bounded repetition with different bounds, D2 requires the question-mark repetition, S1 requires single-bounded repetition, S3 requires double-bounded repetition with the same bounds,  L1 requires a lower-bound repetition, L2 requires the kleene star (\verb!*!) repetition, L3 requires the add (\verb!+!) repetition, and C3 requires a negated custom character class.

\subsubsection{Features  and Pattern}
For some representations, the presence of a feature is not enough to determine membership.
%However,  the presence of a feature and properties of the pattern can determine membership.
Identifying D3 requires an OR containing at least two entries with a sequence present in one entry repeated N times, and then the same sequence present in another entry repeated N+1 times.  This is a hard pattern to detect directly, but we identified candidates by looking for a sequence of N repeating groups with an OR-bar (ie. \verb!|!) next to them on one side (either side).  This produced a list of 113 candidates which we narrowed down manually to 10 actual members.


Identifying T2 requires a literal feature that matches the regex \verb!(\\x[a-f0-9A-F]{2})! which reliably identifies hex codes within a pattern.
Similarly T4 requires a literal feature and must match the regex \verb!((\\0\d*)|(\\\d{3}))! which is specific to Python-style octal, requiring either exactly three digits after a slash, or a zero and some other digits after a slash.  Only one false positive was identified which was actually the lower end of a hex range using the literal \verb!\0!.

Identifying T3 requires that a single literal character is wrapped in a custom character class (a member of T3 is always a member of C2).
 T1 requires that no characters are wrapped in brackets or are hex or octal characters, which actually matches over 91\% of the total patterns analyzed.

\subsubsection{Token Stream }
The rest of the representations were identified by representing the regex patterns as a sequence of tokens.
Identifying S2 requires any element to be repeated at least twice in sequence. This element could be a character class, a literal, or a collection of things encapsulated in parentheses.
Identifying C1 requires that a non-negative character class contains a range.  Identifying C2 requires that there exists a custom character class that does not use ranges or defaults. Identifying C4 requires the presence of a default character class within a custom character class, specifically, \verb!\d!, \verb!\D!, \verb!\w!, \verb!\W!, \verb!\s!, \verb!\S! and \verb!.!.  Identifying C5 requires an OR of length-one sequences (literal characters or any character class).


\subsection{Results}
Table~\ref{table:nodeCount} presents the frequencies with which each representation appears in a regex pattern and in a project scraped from GitHub. The \emph{node} column references the representations in Figure~\ref{fig:refactoringTree} and the \emph{description} column briefly describes the representation, followed by an \emph{example} from the corpus. The \emph{nPatterns} column counts the patterns that belong to the representation, followed by the percent of patterns out of 13,597.
The \emph{nProjects} column counts the projects that contain a regex belonging to the representation,
followed by the percentage of projects out of 1,544.
Recall that the patterns are all unique and could appear in multiple projects, hence the project support is used to show how pervasive the representation in across the whole community.
For example, 2,479 of the patterns belong to the C1 representation, representing 18.2\% of the patterns. These appear in 810 projects, representing 52.5\%.
 Representation D1 appears in 346 (2.5\%) of the patterns but only 234 (15.2\%) of the projects. In contrast, representation T3 appears in 39 \emph{fewer} patterns but 34 \emph{more} projects, indicating that D1 is more concentrated in a few projects and T3 is more widespread across projects.

Using the pattern frequency as a guide, we can create refactoring recommendations based on community frequency. For example, since C1 is more prevalent than C2 in both patterns and projects, we could say that C2 is smelly since it could better conform to the community standard if expressed as C1. Thus, we might recommend a $\overrightarrow{C2C1}$ refactoring. Based on patterns alone, the winning representations per equivalence class are C1, D2, T1, L2, and S2. With one exception, these are the same for recommendations based on projects. The difference is that L3 appears in more projects than L2, so it is not clear which desirable based on community standards metrics.
However, we note that our criteria for membership in a representation may overestimate the opportunities for refactoring. For example, \verb![a-f]! in C1 cannot be refactored to C4 since there does not exist a default character class for that range of characters. A finer-grained analysis is needed to identify actual refactoring opportunities. Our analysis simply suggests a direction for a refactoring (in this case, from C4 to C1). 
%Section~\ref{sec:rq3} explores these results more deeply.

%Table~\ref{summaryResults} presents these recommendations for each pair of representations within each equivalence class. The \emph{Comm} column is populated based on the findings of \emph{RQ1}. The findings for \emph{RQ2} and \emph{RQ3} are in the \emph{Match} and \emph{Compose} columns, respectively.



\section{Desirable Representations (RQ3)}
\label{sec:rq3}
To determine the overall trends in the data, we created and compared total orderings on the representation nodes in each equivalence class (Figure~\ref{fig:refactoringTree})  with respect to the community standards (RQ1)  and understandability (RQ2) metrics.

\subsection{Analysis}
At a high level, the total orderings were achieved by building directed graphs with the representations as nodes and edge directions determined by the metrics: patterns and projects for community standards and matching and composition for understandability. Then, within each graph, we performed a topological sort to obtain total node orderings.

The graphs for community support are based on Table~\ref{table:nodeCount} and the graphs for understandability are based on Table~\ref{table:testedEdgesTable}. The following sections describe the processes for building and and topologically sorting the graphs. 

\begin{figure}[tb]
\centering
\includegraphics[width=0.42\columnwidth]{graphs/cart.pdf}\includegraphics[width=0.57\columnwidth]{graphs/ccom.pdf}
\vspace{-12pt}
\caption{Trend graphs for the CCC equivalence graph: (a) represent the artifact analysis (RQ2), (b) represent the understandability analysis (RQ1).}

\label{fig:graphsforanalysis}
\end{figure}


%\begin{table}
%\centering
%\caption{Topological Sorting, with the left-most position being highest \label{topologicalResults}}
%\begin{footnotesize}
%\begin{tabular}{| l | l | l | l | l | l |} \hline
%				& CCC			& DBB 		& LBW & SNG & LIT \\ \hline
%U 			& C1 C5 C4 C2 C3 	& D3 D1 D2 	& L3 L2		& S2 S1		& T1 T2 T4 T3 \\
%C		& C1 C3 C2 C4 C5 	& D2 D1 D3	&  L3 L2 L1 	& S2 S1 S3 	& T1 T3 T2 T4 \\
%\hline
%\end{tabular}
%\end{footnotesize}
%\end{table}

\begin{table}
\centering
\caption{Topological Sorting, with the left-most position being highest \label{topologicalResults}}
\vspace{-6pt}
\begin{tabular}{| l | l | l |}  \hline
& Understandability & Community  \\ \hline 
CCC & C1 C5 C4 C2 C3  &   C1 C3 C2 C4 C5  \\
DBB & D3 D1 D2  &   D2 D1 D3\\
 LBW & L3 L2	 &  L3 L2 L1 	\\
 SNG &  S2 S1 &  S2 S1 S3 \\
 LIT & T1 T2 T4 T3 & T1 T3 T2 T4 \\
\hline
\end{tabular}
\end{table}


\subsubsection{Building the Graphs}
In the community standards graph, we represent a directed edge  $\overrightarrow{C2  C1}$ when  nPatterns(C1) $>$ nPatterns(C2) \emph{and}  nProjects(C1) $>$ nProjects(C2).
When there is a conflict between nPatterns and nProjects, as is the case between L2 and L3, 
%where L2 is found in more patterns and L3 is found in more projects, 
an undirected edge $\overline{L2L3}$ is used as there is no winner based on the  metrics. 
After considering all pairs of nodes in each equivalence class that also have an edge in Figure~\ref{fig:refactoringTree}, we create graphs, for example Figure~\ref{fig:graphsforanalysis}a, that represents the frequency trends among the community artifacts. A node with no incoming edges is less common and a node with many incoming edges is more common. 
%Note that with the CCC group, there is no edge between C3 and C5 because there is no straightforward refactoring between those representations, as discussed in Section~\ref{sec:refactoring}.

In the understandability graph, we represent a directed edge  $\overrightarrow{C2C1}$ when match(C1) $>$ match(C2) \emph{and} compose(C1) $>$ compose(C2). When there is a conflict between match and compose, as is the case with T1 and T3 where match(T1) is higher but compose(T3) is higher, an undirected edge $\overline{T1T3}$ is used. When one metric has a tie, as is the case with composition in E9, we use the other metric to determine  $\overrightarrow{C5C1}$. An example understandability graph for the CCC is shown in Figure~\ref{fig:graphsforanalysis}b. Nodes with few incoming edges are less understandable (or were not evaluated in our study), and nodes with more incoming edges were more understandable. 
%\footnote{When there are confounded representations, as is the case with E8, E4, and E5 which all use tranformations from the CCC and the LIT equivalence classes, we omit those from the understandability graph. This makes sense since all use a transformation between T1 and T4 strongly favoring T1. }

\subsubsection{Topological Sorting}
Once the graphs are built for each equivalence class and each set of metrics, we apply a modified version of Kahn's topological sorting algorithm to obtain a total ordering. 
%The first modification is to remove all undirected edges since Kahn's operates over a directed graph. 
%To begin, any disconnected nodes are added to the end of the topologically sorted list $L$. 
%In Kahn's algorithm, all nodes without incoming edges are added to a set $S$, which represents the order in which nodes are explored in the graph. For each $n$ node in $S$, all edges from $n$ are removed and $n$ is added to a list $L$. If there exists a node $m$ that has no incoming edges, it is added to $S$.  In the end, $L$ is a topologically sorted list.
%\begin{algorithm}
%  \caption{Modified Topological Sort}\label{topological}
%  \begin{algorithmic}[1]
%\State  $L \gets$ []
%\State $S \gets$ []
%\State Remove all undirected edges (creates a DAG) \label{removeundir}
%\State Add all disconnected nodes to $L$ and remove from graph. If there is more than one, mark the tie. \label{markTie1}
%\State Add all nodes with no incoming edges to $S$. If there is more than one, mark the tie. \label{addnoincomingtos}
%\While {$S$ is non-empty} \label{beginwhile}
%	\State remove a node $n$ from $S$ \label{setn}
%	\State add $n$ to $L$  \label{addntoL}
%	\For {node $m$ such that $e$ is an edge $\overrightarrow{nm}$}
%		\State remove $e$
%		\If{$m$ has no incoming edges}
%			\State add $m$ to $S$ \label{addToS}
%		\EndIf
%	\EndFor
%	\State If multiple nodes were added to $S$ in this iteration, mark the tie \label{markTie2}
%	\State remove $n$ from graph
%\EndWhile
%\State For all ties in $L$, use a tiebreaker.
%  \end{algorithmic}
%\end{algorithm}
One downside to Kahn's algorithm is that the total ordering is not unique and often multiple nodes with similar properties (e.g., no incoming edges) could be considered tied. Thus, we mark ties in order to identify when a tiebreaker is needed to enforce a total ordering on the nodes (though admittedly, it is not always unique). 
%For example, on the understandability graph in Figure~\ref{fig:graphsforanalysis}, there is a tie between C3 and C2 since both have no incoming edges, so they are marked as a tie. Further, if C3 is added to $S$ first, when $n=C2$, both C5 and C4 are added to $S$, thus the tie between them is marked. In these cases, a tiebreaker is needed.
Breaking ties on the community standards graph involves choosing the representation that appears in a larger number of projects, as it is more widespread across the community. 
Breaking ties in the understandability graph uses the metrics. Based on Table~\ref{table:testedEdgesTable}, we compute the average matching score for all instances of each representation, and do the same for the composition score. For example, C4 appears in E5, E12 and E13 with an overall average matching score of 0.81 and composition score of 24.3. C5 appears in E4 and E9 with an average matching of 0.87 and composition of 28.28. Thus, C5 is favored to C4 and appears higher in the sorting.

\subsection{Results}
After running the topological sort with tiebreakers, we have a total ordering on nodes for each graph, shown in Table~\ref{topologicalResults}.  For example, given the graphs in Figure~\ref{fig:graphsforanalysis}a and Figure~\ref{fig:graphsforanalysis}b, the topological sorts are {\tt C1 C3 C2 C4 C5} and {\tt C1 C5 C4 C2 C3}, respectively.



Considering both topological sorts, there is a clear winner in each equivalence class, with the exception of DBB.
%That is, the node sorted highest in the topological sorts for both the community standards and understandability analyses are 
This is C1 for CCC, L3 for LBW, S2 for SNG, and T1 for LIT.
After the top rank, the second rank varies depending on the metric, however, having a consistent and clear winner is evidence of a preference with respect to community standards and understandability, and thus provides guidance for potential refactorings.

This positive result, that the most popular representation in the corpus is also the most understandable, makes sense as people may be more likely to understand things that are familiar or well documented. However, while L3 is the winner for the LBW group, we note that L2 appears in slightly more patterns.
DBB is different  as the orderings are completely reversed depending on the analysis, so the community standards favor D2 and understandability favors D3. Further study is needed on this, as well as  LBW and SNG since not all nodes were considered in the understandability analysis. 





\input{results}
\input{table/groupANOVATable}
\section{Discussion}
\label{sec:discussion}
Based on our analyses of source code and our empirical study on the understandability of regex representations, we have identified preferred regex representations that may make regexes easier to understand and thus maintain. In this section, we describe the implications of these results.

\subsection{Interpreting Results}
In the CCC equivalence class, C1 (e.g., \verb![0-9a]!) is more commonly found in the patterns and projects.  Representations C2 (e.g., \verb![0123456789a]!) and C3 (e.g., \verb![^\x00-/:-`b-\x7F]!) appear in similar percentages of patterns and projects but there is no significant difference in understandability considering two pairs of regexes tested as part of E13 (Table~\ref{table:testedEdgesTable}). However, a small preference is shown for C1 over C2 (E7), leading this to to be the winner of both the community support and understandability analyses.
%    Regex length is probably important for understandability, though we did not test for this.

% - the longest regex in the corpus is \todoNow{X} characters long...
%Anyway C2 is comon but less readable.  C4 is somehwat less common to use defalut in CC - why?  C5 is rare, but marginally more readable than C2.  Not enough data or contrast to come to a conclusion about C3 - it is a catch-all?

In the DBB group, D3 (e.g., \verb!pBs|pBBs|pBBBs!) merits further exploration because it is the most understandable but least common node in DBB group.  This may be because explicitly listing the possibilities with an OR is easy to grasp, but if the number of items in the OR is too large, the understandability may go down. Further analysis is needed to determine the optimal thresholds for representing a regex as D3 compared to D1 (e.g., \verb!pB{1,3}s!) or D2 (e.g., \verb!pBB?B?s!).
%Intuitively, it seems that D2 may be more common because 0,1 is just a more common use case than an arbitrary range like 4, 25.

In the SNG group, S1 is a compact representation (e.g., \verb!S{3}!), but S2 was preferred (e.g., \verb!SSS!). Similar to the DBB group, this may be do to the particular examples chosen in the analysis, as a large number of explicit repetitions may not be as preferred.

In the LWB group,  representations L2 (e.g., \verb!AAA*!) and L3 (e.g., \verb!AA+!) appear in similar numbers of patterns and projects, but there is a significant difference in their understandability, favoring L3.
% , it's clear that this is a rare use case, and also that L3 is the most common  use case.  Patterns using star are secondary, helper patterns because they will trivially match anything, so they are less common.  But anyway...


% is nice, but
%so probably better than S2.  S2 is over-weighted because of double-characters in regular words like foot.
In the LIT group, T1 (e.g., \verb!\a\$>!) is the typical way to list literals, but the reason to use hex (T2) or oct (T4) types is because some characters cannot be represented any other way, such as invisible chars.  One main result of our work is that  T4 (e.g., \verb!\007\036\062!) is  less understandable   than T2 (e.g., \verb!\x07\x24\x3E!), so if invisible chars are required, hex is the more understandable representation.
Regarding T3 (e.g., \verb!\a[$]>!), initially we thought the square brackets would be more understandable than using an escape character,  but we found the opposite. Given a choice between T1 and T3, the escape character was more understandable.

\subsection{Opportunities For Future Work}
There are several directions for future work related to regex study and refactoring.

\paragraph{Equivalence Class Models}
\label{sec:futureequivclasses}
We looked at five equivalence classes, each with three to five nodes.
Future work could consider richer models with more or different classes and nodes.
%For example, we have looked at all ranges as equivalent, all defaults as equivalent, and relied on many such generalizations.  However, the range \verb![a-f]! is likely to be more understandable for most people than a range like \verb![:-`]!.
%In addition to breaking our 5 groups into more specific nodes, future work could model refactorings outside of these groups.
%
%We have not determined a list of all possible refactoring groups given the functional variety and significant number of features to consider, but we are aware of a few additional equivalence classes outside of our 5 groups, such as:
Additional equivalence groups to consider may include:
\begin{itemize} \itemsep -2pt
%\item[Single line option]  \verb!'''(.|\n)+'''! $\equiv$ \verb!(?s)'''(.)+'''!
\item Multi line option:  \verb!(?m)G\n! $\equiv$ \verb!(?m)G$!
\item Case insensitive:  \verb!(?i)[a-z]! $\equiv$ \verb![A-Za-z]!
\item Backreferences:  \verb!(X)q\1! $\equiv$ \verb!(?P<name>X)q\g<name>!
%\item[Word Boundaries]  \verb!\bZ! $\equiv$ \verb!((?<=\w)(?=\W)|(?<=\W)(?=\w))Z!
\end{itemize}



It might also be the case that there exist critical comprehension differences within a representation. For example, between C1 (e.g., \verb![0-9a]!) and C4 (e.g., \verb![\da]!), it could be the case that \verb![0-9]! is preferred to \verb![\d]!, but \verb![A-Za-z0-9_]! is not be preferred to \verb![\w]!.
By creating a more granular model of equivalence classes  and carefully evaluating alternative representations of the most frequently used specific patterns,  additional useful refactorings could be identified.


%We focused on refactorings within a group, treating groups as orthogonal to one another.  It would be interesting to see if there is some cooperation between pairs of edges in separate groups by applying more than one type of refactoring at once.

%\paragraph{Understandability}
%
%
%%One of the most straightforward ways to address understandability is to directly ask software professionals which from a list of equivalent regexes they prefer and why.
%% but at least one side of the refactoring was contrived and we did not focus on any specific community (the 1544 projects we obtained regexes from were randomly obtained).
%% If understandability measurements used regexes sampled from the codebase of a specific community(most frequently observed regexes, most buggy regexes, regexes on the hottest execution paths, etc.), and measured the understanding of programming professionals working in that community, then the measurements and the refactorings they imply would be more likely to have a direct and certain positive impact.
%
%In another study, we did a survey where software professionals indicated that understandability of regexes they find in source code is a major pain point.  In this study, our participants indicated that they read about twice as many regexes as they compose.  What is the impact on maintainers, developers and contributors to open-source projects of not being able to understand a regex that they find in the code they are working with?  Presumably this is a frustrating experience - how much does a confusing regex slow down a software professional?  What bugs or other negative factors can be attributed to or associated with regexes that are difficult to understand?  How often does this happen and in what settings?  Future work could tailor an in-depth exploration of the overall costs of confusing regexes and the potential benefits of refactoring or other treatments for confusing regexes.

\vspace{-1pt}
\paragraph{Regex Migration Libraries}
We have identified opportunities
 to improve the understandability of regexes in existing code bases by looking for some of the less understandable regex representations, which can be thought of as antipatterns, and refactoring to the more common or understandable representations.
 Building migration libraries is a promising direction of future work to ease the manual burden of this process, similar in spirit to prior work on class library migration~\cite{Balaban:2005:RSC:1103845.1094832}.

%\paragraph{Regex Refactoring Applications}
%Maintainers of code that is intentionally obfuscated for security purposes may want to develop regexes that they understand and then automatically transform them into the least understandable regex possible.
%
%One fundamental concept that many users of regex struggle to learn is when to use regexes for simple parsing, and when to write a full-fledged parser (for example, when parsing HTML).  Regexes that are trying to parse HTML, XML or similar languages could be refactored not into a better regex, but into some code with an equivalent intention that does parsing much better.

\vspace{-1pt}
\paragraph{Regex Programming Standards}
Many organizations enforce coding standards in their repositories to ease understandability.
Presently, we are not aware of coding standards for regular expressions, but this work suggests that enforcing standard representations for various regex constructs could ease comprehension.

\vspace{-1pt}
\paragraph{Regex Refactoring for Performance}
The representation of regexes may have a strong impact on the runtime performance of a chosen regex engine. Prior work has sought to expedite the processing of regexes over large bodies of text~\cite{Baeza-Yates:1996:FTS:235809.235810}.
Refactoring regexes for performance would complement those efforts.
%Further study is needed to determine which representations are most efficient, leading to a whole new area of study on regex refactoring for performance, a topic already explored for
%Depending on the efficiency of an organization's chosen regex engine, an organization may want to enforce standards for efficiency.
%, or for compatibility with a regex analysis tool like Z3, HAMPI, BRICS or REX.

\subsection{Threats to Validity}

\textbf{Internal:}
We measure understandability of regexes using two metrics, matching and composition. However, these measures may not reflect actual understanding of the regex behavior. We chose to use multiple metrics in the context of reading and writing regexes, but the threat remains.

Participants evaluated regular expressions during tasks on MTurk, which may not be representative enough of the context in which programmers would encounter regexes in practive. Further study is needed to determine the impact of the experimentation context on the results.

Some regex representations from the equivalence classes were not involved in the understandability analysis and that may have biased the results against those nodes. Repetition of the analysis with more compete coverage of the edges in the equivalence classes is needed.

%We treated unsure responses as omissions that  did not count  against the matching scores. Thus, if a participant answered two strings correctly and marked the other three strings as unsure, then this was 2/2 correct, not 2/5. This may have inflated the matching scores, however, less than 5\% of the matching scores were impacted by such responses.



%
%In our analyses, we measure understandability using matching and composition metrics.
%However, there may be other ways to approach regex understandability, such as deciding which regexes in a set are equivalent, finding the minimum modification to some text so that a given regex will match it.
%It may also be meaningful to provide some code that exists around a regex as context, since that would better represent a scenario in which programmers would encounter regexes in practice.
%Further study is needed to determine if the chosen metrics and experimentation context have resulted in a reasonable measure of understandability.

\textbf{External}
Participants in our survey came from MTurk, which may not be representative of people who read and write regexes on a regular basis.

The regexes  used in the evaluation were inspired by those found in Python code, which is just one language that has library support for regexes. Thus, we may have missed opportunities for other refactorings based on how programmers use regexes in other programming languages.

The results of the understandability analysis may be closely tied to the particular regexes chosen for the experiment. For many of the representations, we had several comparisons. Still, replication with more regex patterns is needed.% to validate our results.

%\todoMid{what about the threat of too few examples per node?  Didn't cover every edge.  Regex set is randomly collected online, not focused on any specific target audience.}

%Our community analysis only focuses on the Python language, but as the vast majority of regex features are shared across most general programming languages (e.g., Java, C, C\#, or Ruby), a Python {pattern} will (almost always) behave the same when used in other languages and our results are likely to generalize.
%, whereas a utilization is not universal in the same way (i.e., it may not compile in other languages, even with small modifications to function and flag names).
%As an example, the {\tt re.MULTILINE} flag, or similar, is present in Python, Java, and C\#, but  the Python {\tt re.DOTALL} flag is not present in C\# though it has an equivalent flag in Java.


\section{Related Work}
\label{sec:related}
Regular expression understandability has not been studied directly, though prior work has suggested that regexes are hard to read and understand as there are tens of thousands of bug reports related to regexes~\cite{Spishak:2012:TSR:2318202.2318207}. 
To aid in regex creation and understanding,  tools have been developed to support more robust creation~\cite{Spishak:2012:TSR:2318202.2318207} or to allow visual debugging~\cite{Beck:2014:RVD:2591062.2591111}. Other research has focused on removing the human from the creation process by learning regular expressions from  text~\cite{Babbar:2010:CBA:1871840.1871848, Li:2008:REL:1613715.1613719}.

Code smells in object-oriented languages were introduced by Fowler~\cite{Fowl1999}. Researchers have studied the impact of code smells on program comprehension~\cite{abbes2011empirical, du2006does}, finding that the more smells in the code, the harder the comprehension. 
%This is similar to our work, except we aim to identify which  regex representations can be considered smelly. 
Code smells have been extended to other language paradigms including end-user programming languages~\cite{Hermans2012intra, Hermans2012intraExt, stoleeicse, stoleeTSE}. The code smells identified in this work are representations that are not common or not well understood by developers. Using community standards to define smells has been used in refactoring   for end-user programmers~\cite{stoleeicse, stoleeTSE}. 

Regular expression refactoring has not been studied directly, though refactoring literature abounds~\cite{Mens:2004:SSR:972215.972286, Opdyke:1992:ROF:169783, Griswold:1993:AAP:152388.152389}. 
The closest to regex refactoring comes from research toward  expediting  regular expressions processing on large bodies of text~\cite{Baeza-Yates:1996:FTS:235809.235810},  similar to refactoring for performance. 


Exploring language feature usage by mining source code has been studied extensively for
Smalltalk~\cite{Callau:2011:DUD:1985441.1985448},
JavaScript~\cite{Richards:2010:ADB:1809028.1806598},
and Java~\cite{Dyer:2014:MBA:2568225.2568295, Grechanik:2010:EIL:1852786.1852801, Parnin:2013:AUJ:2589712.2589717, Livshits:2005:RAJ:2099708.2099724},
and more specifically,
Java generics~\cite{Parnin:2013:AUJ:2589712.2589717} and
Java reflection~\cite{Livshits:2005:RAJ:2099708.2099724}.
Our prior work~\cite{chapman2016} was the first to mine and evaluate regular expression usages from software repositories. 
The intention of the prior work was to explore regex language features  usage and surveyed developers about regex usage. 
%In this work, we define potential refactorings and use the mined corpus to find support for the presence of various regex representations before performing an understandability analysis.
% . Beyond that, we measure regex understandability and suggest canonical representations for regexes to enhance conformance to community standards and understandability. 

%Related to mining work, regular expressions have been used to form queries in mining framework~\cite{Begel:2010:CDE:1806799.1806821}, but have not been the focus of the mining activities.
%Surveys have been used to measure adoption of various programming languages~\cite{Meyerovich:2013:EAP:2509136.2509515, Dattero:2004:PLG:962081.962087}, and been combined with  repository analysis~\cite{Meyerovich:2013:EAP:2509136.2509515}, but have not focused on regexes.

%Mining properties of open source repositories is a well-studied topic, focusing, for example, on API usage patterns~\cite{Linares-Vasquez:2014:MEA:2597073.2597085}.

%
%
%
%Refactoring comprehension 
%
%Refactoring for understandability~\cite{6405299, stoleeicse, stoleeTSE} and conformance to the community in end-user programs~\cite{stoleeicse, stoleeTSE}.
%
%
%Regular expression education
%
%
%Focus on language design for a regex extension to Haskell~\cite{Broberg:2004:REP:1016848.1016863}.  
%

%Regular expressions have been a focus point in a variety of research objectives. 
%
%Regarding applications, regular expressions have been used for test case generation~\cite{Ghosh:2013:JAT:2486788.2486925, Galler:2014:STD:2683035.2683100, Anand:2013:OSM:2503903.2503991, Tillmann:2014:TAT:2642937.2642941},  and
%as specifications for string constraint solvers~\cite{Trinh:2014:SSS:2660267.2660372, hampi}.
%
%
%
%As a query language, lightweight regular expressions are pervasive in search. For example,
%some data mining frameworks use regular expressions as queries (e.g., ~\cite{Begel:2010:CDE:1806799.1806821}). 

%One common misconception is that all regular expression languages are \emph{regular languages} which can be represented using deterministic finite automata (DFA), and so they are easy to model, easy to describe formally and execute in O(n) time.  In fact, many regular expression matching engines run in exponential time in order to support useful features such as lazy quantifiers, capturing groups, look-aheads and back-references~\cite{msdnmatching}.  In a recent regular expression library, the RE2 projext~\cite{re2}, Russ Cox aimed to use DFAs as much as possible (maximizing speed) while supporting as many useful features as possible.

%Thousands of research papers have focused on various other regular expression-related investigations.



%In this work, we perform a feature analysis on regular expressions used in the wild and compare that set to the features supported by four popular regular expression tools.
%Research tools like Hampi~\cite{hampi}, and Rex~\cite{rex}, and commercial tools like brics\cite{brics} and RE2~\cite{re2}, all support the use of regular expressions in various ways. Hampi was developed  in academia and uses regular expressions as a specification language for a constraint solver. Rex was developed by Microsoft Research and generates strings for regular expressions that can be used in  applications such as test case generation~\cite{Anand:2013:OSM:2503903.2503991, Tillmann:2014:TAT:2642937.2642941}. Brics is an open-source package that creates automata from regular expressions for manipulation and evaluation.
%RE2 is an open-source tool created by Google to power code search with an efficient regex engine.


%Mining properties of open source repositories is a well-studied topic, focusing, for example, on API usage patterns~\cite{Linares-Vasquez:2014:MEA:2597073.2597085} and bug characterizations~\cite{Chen:2014:ESD:2597073.2597108}.
%Exploring language feature usage by mining source code has been studied extensively for
%Smalltalk~\cite{Callau:2011:DUD:1985441.1985448, Callau:2013:DUD:2589712.2589718},
%JavaScript~\cite{Richards:2010:ADB:1809028.1806598},
%and Java~\cite{Dyer:2014:MBA:2568225.2568295, Grechanik:2010:EIL:1852786.1852801, Parnin:2013:AUJ:2589712.2589717, Livshits:2005:RAJ:2099708.2099724},
%and more specifically,
%Java generics~\cite{Parnin:2013:AUJ:2589712.2589717} and
%Java reflection~\cite{Livshits:2005:RAJ:2099708.2099724}.
%To our knowledge, this is the first work to mine and evaluate regular expression usages from existing software repositories. Related to mining work, regular expressions have been used to form queries in mining framework~\cite{Begel:2010:CDE:1806799.1806821}, but have not been the focus of the mining activities.
%Surveys have been used to measure adoption of various programming languages~\cite{Meyerovich:2013:EAP:2509136.2509515, Dattero:2004:PLG:962081.962087}, and been combined with  repository analysis~\cite{Meyerovich:2013:EAP:2509136.2509515}, but have not focused on regexes.


% \subsection{Research on Regular Expressions}
% Visual debugging of regular expressions~\cite{Beck:2014:RVD:2591062.2591111}

% %the related work section in the Spishak section is very good re: regex tools like those that represent regexes as automata or grammars
% Static analysis to reduce errors in building regular expressions by using a type system to identify errors like {\tt PatternSyntaxExceptions} and {\tt IndexOutOfBoundsExceptions} at compile time~\cite{Spishak:2012:TSR:2318202.2318207}.

% \subsection{Research on Regular Expressions}
% Visual debugging of regular expressions~\cite{Beck:2014:RVD:2591062.2591111}

% \subsection{Research that Depends on Regular Expression Usage}
% Regular expressions are used as queries in a data mining framework~\cite{Begel:2010:CDE:1806799.1806821}


%
%\todoMid{We are building on the survey that indicates regexes are hard to read, and the apparent lack of any regex readability refactoring attempts.  Many papers have talked about refactoring, basically it is changing the form but not the behavior.}


%Prior work that surveyed developers about regex usage found that in a small software company, the 18 surveyed developers compose an average of 172 regexes per year. This is 48\% higher than the number of regexes composed annually by MTurk participants in this work, which may be due to the nature of the jobs performed by the two populations. 
%




\section{Conclusion}
In an effort to find smells that impact regex understandability, we created five equivalence class models and used these models to investigate the most common representations and most comprehensible representations per class.  
%We found the most common representations per class by both number of patterns and number of projects to be C1, D2, T1 and S2 (L3 has the most patterns, L2 has the most projects).
%We  identified three strongly preferred transformations between representations (i.e., $\overrightarrow{T4 T1}$, $\overrightarrow{D2 D3}$, and  $\overrightarrow{L2 L3}$). 
 The high agreement between the community standards and understandability analyses  suggests that one particular representation can be preferred over others in most cases.  
Based on these results, we recommend using hex to represent invisible characters in regexes instead of octal, and to escape special characters with slashes instead of wrapping them in brackets.  
Further research is needed into more granular models that treat common specific cases separately.
%, and that address the effect of length on understandability when transforming from one representation to another.








\balance

\section*{Acknowledgements}
Redacted.  
%This work is supported in part by  NSF SHF-EAGER-1446932.


\bibliographystyle{IEEEtran}
\bibliography{biblio,stolee}

\end{document}

